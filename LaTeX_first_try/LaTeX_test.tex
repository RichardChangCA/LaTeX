\documentclass[11pt]{article}

\usepackage[margin=1in, paperwidth=8.5in, paperheight=11in]{geometry}
\usepackage{amsfonts}
\usepackage{graphicx}

\def\eq1{y=\frac{x}{3x^2+x+1}}
\def\labelaxes{Something unimportant}

\begin{document}

\tableofcontents

\title{\LaTeX \ demo}
\author{Lingfeng Zhang}
\date{\today}
\maketitle

Graph$\eq1$ \labelaxes

the set of natural numbers is denoted by $\mathbb{N}$

the set of Integers is denoted by $\mathbb{Z}$

the set of real number is denoted by $\mathbb{R}$

This is  Bo Li \\
\begin{center}\includegraphics[width=3in]{viva.jpeg}\end{center}
\begin{center}\includegraphics[scale=0.5]{viva.jpeg}\end{center}
\begin{center}\includegraphics[angle=45,scale=0.1]{viva.jpeg}\end{center}

Mention about the image file type

Suppose we are given a rectangle with side lengths $(x+1)$ and$(x+3)$

superscript: $2x^3$,next line: $$2x^{34}$$
$$2x^{3x+4}$$
$$2x^{3x^2+4}$$

subscript:$$x_1$$
$$x_{21}$$
$${x_1}_2$$
$${{x_1}_2}_3$$

Greek letters:
$$\pi$$
$$\alpha$$
$$A=\pi r^2$$

triangle letters:
$$y=\sin{x}$$

log function:
$$\log{x}$$
$$\log_5{x}$$
$$\ln{x}$$

square roots:
$$\sqrt{2}$$
$$\sqrt[3]{2}$$
$$\sqrt{x^2+y^2}$$
$$\sqrt{1+\sqrt{x}}$$

fraction:\\
about 2/3 of the glass is full
$2/3$ nothing changed
$$\frac{2}{3}$$ works
$$\displaystyle{\frac{2}{3}}$$ normal size 
$$\frac{x+1}{x^2+\sqrt{x}}$$
$$\frac{\sqrt{x+1}}{\sqrt{x-1}}$$
$$\frac{1+\frac{2}{3}}{4}$$
$$\sqrt{\frac{x+1}{x^2+3}}$$

brackets:
$$(x+1)$$
$$[x+2]$$
$$\{x+3\}$$
$$\$12.55$$
$$3(\frac{2}{3})$$parameters too small
$$3\left(\frac{2}{3}\right)$$
$$3\left[\frac{2}{3}\right]$$
$$3\left\{\frac{2}{3}\right\}$$
$$|x|$$
$$\left|\frac{x}{2+1}\right|$$
$$\left\{x^2\right\}$$
missing right curly parameter by using dot .$$\left\{x^2\right.$$ 
$$\left|\frac{dy}{dx}\right|_{x+1}$$
$$\left.\frac{dy}{dx}\right|_{x+1}$$

table:

\begin{tabular}{ccccccc}
$x$ &1&2&3&4&5&6\\ 
$f(x)$ &11&12&13&14&15&16\\ 
\end{tabular}

\begin{tabular}{ccccccc}
\hline
$x$ &1&2&3&4&5&6\\ \hline
$f(x)$ &11&12&13&14&15&16\\ \hline
\end{tabular}

\begin{tabular}{|c|c|c|c|c|c|c}
\hline
$x$ &1&2&3&4&5&6\\ \hline
$f(x)$ &11&12&13&14&15&16\\ \hline
\end{tabular}

equation array with (sequence number)
\begin{eqnarray}
5x^2-9=x+3\\
x^4 = 4+x\\
x +4 = 3\\
x\approx\pm1.123
\end{eqnarray}
\\
equation array without {sequence number}
\begin{eqnarray*}
5x^2-9=x+3\\
x^4 = 4+x\\
x +4 = 3\\
x\approx\pm1.123
\end{eqnarray*}
equation array with {sequence number} and align by "="
\begin{eqnarray}
5x^2-9&=&x+3\\
x^4 &=& 4+x\\
x +4 &=& 3\\
x&\approx&\pm1.123
\end{eqnarray}

list:

\begin{enumerate}
\item pencil
\item calculator
\item ruler
\item notebook
	\begin{enumerate}
	\item assessments
		\begin{enumerate}
		\item quizzes
		\item tests
		\end{enumerate}
	\item homework
	\item notes $\sigma\theta$
	\end{enumerate}
\item graph paper
\end{enumerate}

\begin{itemize}
\item pencil
\item calculator
\item ruler
\item notebook
	\begin{itemize}
	\item assessments
		\begin{itemize}
		\item quizzes
		\item tests
		\end{itemize}
	\item homework
	\item notes $\sigma\theta$
	\end{itemize}
\item graph paper
\end{itemize}

label right justified:

\begin{enumerate}
\item[Commutative] $a+b=b+a$
\item[Associative] $a+(b+c)=(a+b)+c$
\item[Distributive] $a(b+c)=ab+ac$
\end{enumerate}

This will produce \textit{italicized} text

This will produce \textbf{bold-faced} text

This will produce \textsc{small caps} text

This will produce \texttt{typewriter} font

Please will my github at \texttt{http://www.github.com/RichardChangCA}.

please call me Lingfeng Zhang

please call me \begin{large}
Lingfeng Zhang
\end{large} $-->$ large

please call me \begin{Large}
Lingfeng Zhang
\end{Large} $-->$ Large

please call me \begin{huge}
Lingfeng Zhang
\end{huge} $-->$ huge

please call me \begin{Huge}
Lingfeng Zhang
\end{Huge} $-->$Huge

please call me \begin{small}
\emph{Lingfeng Zhang}
\end{small}$-->$ small

please call me \begin{tiny}
Lingfeng Zhang
\end{tiny} $-->$tiny

\begin{center}
This is center
\end{center}

\begin{flushleft}
this is left justified
\end{flushleft}

\section{Linear functions}
	\subsection{matrix}
	\subsection{vector}
	\subsection{space}
\section{Quadratic functions}
	\subsection{Standard form}

Using structure and click these icons$\bigtriangledown\amalg\odot$

%comment

\end{document}
